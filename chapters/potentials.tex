%%%%%%%%%%%%%%%%%%%%%%%%%%%%%%%%%%%%%%%%%%%%%%%%%%%%%%%%%%%%%%%%%%%%%%%%%%%%%%%%
%                                                                              %
%            A Painless Introduction to Programming UAMMD Modules              %
%                    Chapter 2: Unleash your own potential                     %
%                                                                              %
%                          Marc Meléndez Schofield                             %
%                                                                              %
%%%%%%%%%%%%%%%%%%%%%%%%%%%%%%%%%%%%%%%%%%%%%%%%%%%%%%%%%%%%%%%%%%%%%%%%%%%%%%%%

You can only have so much fun playing with a Lennard-Jones simulation. Changing
the parameter values and then looking for differences in your plots gets old
soon. If you have made it this far, you want more. You imagine molecules, or
larger structures. Perhaps you would like to simulate billowing smoke, sloshing
water, or planets orbiting a star. All in good time.

Whatever your final aim, you need new interactions, and this chapter contains a
good collection of them.

\section{Harmonic bonds}

The simplest permanent link between particles in UAMMD has to be the harmonic
bond. Think of it as an invisible linear spring connecting two particles. As
long as they remain at the equilibrium distance $r_0$, they exert no force on
each other, but when you stretch their separation $r$, they pull back. The
potential function for this bond, then, equals
\begin{equation*}
  V(r) = \frac{1}{2}\ k\ (r - r_0)^2,
\end{equation*}
where the spring constant $k$ quantifies the bond rigidity.

Take a string of one hundred and one particles as our explanatory device. We
will place them all on a slight curve line and initially at rest.
\begin{lstlisting}
%! codeblock: stringInitialConditions
  int numberOfParticles = 101;
  auto particles
    = make_shared<ParticleData>(numberOfParticles, sys);

  {
    auto position
      = particles->getPos(access::location::cpu,
                          access::mode::write);
    auto velocity
      = particles->getVel(access::location::cpu,
                          access::mode::write);

    real amplitude = 0.1;
    real stringlength = 1.0;
    int modenum = 1;
    for(int i = 0; i < numberOfParticles; ++i) {
      position[i].x = i*(stringlength/(numberOfParticles - 1));
      position[i].y = amplitude*sin(modenum*M_PI*position[i].x);
      position[i].z = position[i].w = 0;
      velocity[i].x = velocity[i].y = velocity[i].z = 0;
    }
  }
%! codeblockend
\end{lstlisting}

UAMMD reads bond information in from an external file. Let me first explain the
format of this plain text bond file. It should look something like this:
\begin{lstlisting}
100
0 1 1000.0 0.01
1 2 1000.0 0.01
2 3 1000.0 0.01
3 4 1000.0 0.01
    . . .
\end{lstlisting}

and so on. The file begins with the number of bonds ($100$) followed by a row
for each bond presenting $i$, $j$, $k$ and $r_0$, that is, the particle ID
numbers, the bond spring constant and the equilibrium distance. Hence, the next
line links particles $0$ and $1$ with a bond of $k = 1000.0$ and $r_0 = 0.01$.
After that, we get the values for the bond connecting particles $1$ and $2$.
Even though I have chosen identical values of $k$ and $r_0$ for all the bonds in
this example, you can of course make them all different if you like.

In addition to linking particles, you can attach them to fixed points in space.
We will pin particle $0$ to the origin of coordinates and particle $100$ to point
$(1, 0, 0)$. When you let go of the string of particles it should vibrate. After
the previous list of bonds, we must include the number of fixed bonds (only $2$
in our case). The format for these is simply $i$, $x$, $y$, $z$, $k$, $r_0$,
meaning: particle ID, position in space $(x, y, z)$, spring constant and
equilibrium distance. So in our example, the bond file should end with:
\begin{lstlisting}
2
0 0 0 0 1000.0 0.0
100 1 0 0 1000.0 0.0
%!
\end{lstlisting}

We will make the program write the bond information so that we don't have to.
\begin{lstlisting}
%! codeblock: stringBondFile
  {
    std::ofstream bondInfo("data.bonds");
    if(not bondInfo.is_open()) {
      sys->log<System::CRITICAL>("Unable to create data.bonds file. Halting program.");
      exit(-1);
    }

    bondInfo<<(numberOfParticles - 1)<<endl;
    for(int i = 0; i < numberOfParticles - 1; ++i) {
      bondInfo<<i<<" "<<(i + 1)<<" 1000.0 0.01"<<endl;
    }
    bondInfo<<"1"<<endl;
    bondInfo<<"0 0 0 0 1000.0 0.0"<<endl;
    bondInfo<<"100 1 0 0 1000.0 0.0"<<endl;
  }
%! codeblockend
\end{lstlisting}

Including the bonds presents no difficulties from the point of view of UAMMD
code, as the code resembles the lines we wrote for the Lennard-Jones
interaction. You insert the customary include at the top,
\begin{lstlisting}
# include "Interactor/BondedForces.cuh"
%!
\end{lstlisting}
define the intereaction parameters to create the ``interactor'',
\begin{lstlisting}
%! codeblock: stringBondInteractor
  {
    using HarmonicBonds = BondedForces<BondType::Harmonic>;
    HarmonicBonds::Parameters bondParameters;
    bondParameters.file = "data.bonds";
    auto bonds = make_shared<HarmonicBonds>(particles, sys, bondParameters);
%! codeblockend
\end{lstlisting}
and then add the bond interaction to the integrator.
\begin{lstlisting}
%! codeblock: stringAddInteractor
    integrator->addInteractor(bonds);
  }
%! codeblockend
\end{lstlisting}

\begin{comment}
Vibrating string simulation code:
\begin{lstlisting}
%! codefile: code/vibrating_string.cu
# include "uammd.cuh"
# include "utils/InitialConditions.cuh"
# include "Interactor/BondedForces.cuh"
# include "Integrator/VerletNVE.cuh"

using namespace uammd;
using std::make_shared;
using std::endl;

int main(int argc, char *argv[]){

  auto sys = make_shared<System>(argc, argv);

  %! codeinsert: stringInitialConditions

  %! codeinsert: VerletParams src: chapters/first_simulation.tex

  %! codeinsert: Verlet src: chapters/first_simulation.tex

  %! codeinsert: stringBondFile

  %! codeinsert: stringBondInteractor

  %! codeinsert: stringAddInteractor

  %! codeinsert: integration src: chapters/first_simulation.tex

  sys->finish();

  return 0;
}
%! codeend
\end{lstlisting}
\end{comment}

\section{External fields}

A straight line of linked particles does not make for much of an illustration,
until you realise that you can activate gravity and let go of one of the ends to
simulate a hanging rope.



\section{Alternative types of bonds}

FENE

Fixed bonds

Angular bonds

Torsional bonds

\section{Other potentials (DPD, SPH)}

\section{Defining your own potentials}

Bond potentials

Pair interactions
