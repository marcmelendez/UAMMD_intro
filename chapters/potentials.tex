You can only have so much fun playing with a Lennard-Jones simulation. Changing
the parameter values and then looking for differences in your plots gets old
fast. If you have made it this far, you want more. You imagine molecules, or
larger structures. Perhaps you would like to simulate billowing smoke, sloshing
water, or planets orbiting a star. All in good time.

Whatever your final aim, you need new interactions, and this chapter contains a
good collection of them.

\section{Harmonic bonds}

The simplest permanent link between particles in UAMMD has to be the harmonic
bond. Think of it as an invisible linear spring connecting two particles. As
long as they remain at the equilibrium distance $r_0$, they exert no force on
each other, but when you stretch their separation $r$ they pull back. The
potential function for this bond, then, equals
\begin{equation*}
  V(r) = \frac{1}{2}\ k\ (r - r_0)^2,
\end{equation*}
where the spring constant $k$ quantifies the bond rigidity.

Including bonds presents no difficulties from the point of view of UAMMD code,
as the code resembles the lines we wrote for the Lennard-Jones interaction.
\begin{lstlisting}
\end{lstlisting}
The difficulty arises from the need to specify which particles are linked.

\section{Alternative types of bonds}

FENE

Fixed bonds

Angular bonds

Torsional bonds

\section{External fields}

\section{Other potentials (DPD, SPH)}

\section{Defining your own potentials}

Bond potentials

Pair interactions
