Let us take the first step of our journey towards high-performance computing on
GPUs with a low-performance simple program that does hardly no computing. We
will modify a typical ``Hello, World!'' program in C++ slightly by adding a few
more lines (marked below with the comment \texttt{/* UAMMD */}).
\begin{lstlisting}[frame=single]
%! codefile: ../code/minimal.cu
# include "uammd.cuh" /* UAMMD */

using namespace uammd; /* UAMMD */
using std::make_shared;
using std::endl;
using std::cout;

int main(int argc, char *argv[]){

  auto sys = make_shared<System>(argc,argv);/* UAMMD */

  cout<<endl<<"--> Hello, UAMMD! <--"<<endl<<endl;

  sys->finish(); /* UAMMD */

  return 0;
}
%! codeend
\end{lstlisting}
Compile the code with
\begin{verbatim}
nvcc -O3 -I../uammd/src -I../uammd/src/third_party -o minimal
minimal.cu
\end{verbatim}
making sure that you replace the paths to UAMMD src code with the right path on
your system. Running \textit{./minimal} outputs some UAMMD information, with
your ``Hello, UAMMD!'' message in the middle. Take out all the lines marked
\texttt{/* UAMMD */} and you are left with a typical ``Hello, World!'' program
in C++ that only outputs your message.

It might be helpful to go through the program to explain what each UAMMD line 
does in very general terms. You \texttt{include} the \texttt{uammd.cuh} header 
file and use the \texttt{uammd} namespace to have access to the functionality 
defined in the UAMMD source code. Within the main function, we enclose all our 
UAMMD code between the creation of a \texttt{System} and its destruction. The
\texttt{System} represents the computing environment in UAMMD. It deals with
access to the GPU and random number generation on the CPU, for example. In the
example above, we created a \texttt{System} named \texttt{sys} with
\begin{verbatim}
auto sys = make_shared<System>(argc,argv);
\end{verbatim}
and destroyed it at the end of the program with \texttt{sys->finish();}.

The program compiles and runs, but it feels like an empty sandwich. By expanding
it slightly, though, we can run a simple simulation.

\section{The physical system}

The main sandwich filling in our virtual kitchen has to be the physical system
that we wish to simulate, represented as a collection of particles.

Free expansion of a gas

\section{The simulation box}

\section{Interactions}

\section{Integrators}

